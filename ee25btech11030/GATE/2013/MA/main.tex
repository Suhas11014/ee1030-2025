\documentclass[journal,12pt,onecolumn]{IEEEtran}
\usepackage{cite}
\usepackage{graphicx}
\usepackage{amsmath,amssymb,amsfonts,amsthm}
\usepackage{algorithmic}
\usepackage{graphicx}
\usepackage{textcomp}
\usepackage{xcolor}
\usepackage{txfonts}
\usepackage{listings}
\usepackage{enumitem}
\usepackage{mathtools}
\usepackage{gensymb}
\usepackage{comment}
\usepackage[breaklinks=true]{hyperref}
\usepackage{tkz-euclide} 
\usepackage{listings}
\usepackage{gvv}                                        
%\def\inputGnumericTable{}                                 
\usepackage[utf8]{inputenc} 
\usetikzlibrary{arrows.meta, positioning}
\usepackage{xparse}
\usepackage{color}                                            
\usepackage{array}                                            
\usepackage{longtable}                                       
\usepackage{calc}                                             
\usepackage{multirow}
\usepackage{multicol}
\usepackage{hhline}                                           
\usepackage{ifthen}                                           
\usepackage{lscape}
\usepackage{tabularx}
\usepackage{array}
\usepackage{float}
\newtheorem{theorem}{Theorem}[section]
\newtheorem{problem}{Problem}
\newtheorem{proposition}{Proposition}[section]
\newtheorem{lemma}{Lemma}[section]
\newtheorem{corollary}[theorem]{Corollary}
\newtheorem{example}{Example}[section]
\newtheorem{definition}[problem]{Definition}
\newcommand{\BEQA}{\begin{eqnarray}}
\newcommand{\EEQA}{\end{eqnarray}}
\usepackage{float}
%\newcommand{\define}{\stackrel{\triangle}{=}}
\theoremstyle{remark}
\usepackage{circuitikz}
\usepackage{tikz}

\title{GATE MA 2013}
\author{EE25BTECH11030-AVANEESH}

\begin{document}

\maketitle

\section*{Q.1 - Q.25 carry one mark each.}
\begin{enumerate}

%1
    \item The possible set of eigen values of a $4 \times 4$ skew-symmetric orthogonal real matrix is
    \begin{enumerate}
    \begin{multicols}{4}
        \item $\{\pm i\}$
        \item $\{\pm i, \pm 1\}$
        \item $\{\pm 1\}$
        \item $\{0, \pm i\}$
        \end{multicols}
    \end{enumerate}
    \hfill (GATE MA 2013)
%2
    \item The coefficient of $(z-\pi)^2$ in the Taylor series expansion of 
    $f(z) = \begin{cases} \frac{\sin z}{z-\pi} & \text{if } z \neq \pi \\ -1 & \text{if } z = \pi \end{cases}$ around $\pi$ is
    \begin{enumerate}
    \begin{multicols}{4}
        \item $\frac{1}{2}$
        \item $-\frac{1}{2}$
        \item $\frac{1}{6}$
        \item $-\frac{1}{6}$
    \end{multicols}
    \end{enumerate}
    \hfill (GATE MA 2013)
%3
    \item Consider $\mathbb{R}^2$ with the usual topology. Which of the following statements are TRUE for all A, B $\subseteq \mathbb{R}^2$? \\
    $P: \overline{A \cup B} = \overline{A} \cup \overline{B}$ \\
    $Q: \overline{A \cap B} = \overline{A} \cap \overline{B}$ \\
    $R: (A \cup B)^{\circ} = A^{\circ} \cup B^{\circ}$ \\
    $S: (A \cap B)^{\circ} = A^{\circ} \cap B^{\circ}$.
    \begin{enumerate}
    \begin{multicols}{4}
        \item P and R only
        \item P and S only
        \item Q and R only
        \item Q and S only
    \end{multicols}
    \end{enumerate}\
    \hfill (GATE MA 2013)
%4
    \item Let $f: \mathbb{R} \rightarrow \mathbb{R}$ be a continuous function with $f(1)=5$ and $f(3)=11$. If $g(x) = \int_{1}^{3} f(x+t) dt$, then $g'(0)$ is equal to \underline{\hspace{1cm}}.
    \hfill (GATE MA 2013)
%5
    \item Let P be a $2 \times 2$ complex matrix such that trace$(P)=1$ and det$(P)=-6$. Then, trace$(P^4 - P^3)$ is \underline{\hspace{1cm}}.
    \hfill (GATE MA 2013)
%6
    \item Suppose that R is a unique factorization domain and that a, $b \in R$ are distinct irreducible elements. Which of the following statements is TRUE?
    \begin{enumerate}
        \item The ideal $\langle 1+a \rangle$ is a prime ideal
        \item The ideal $\langle a+b \rangle$ is a prime ideal
        \item The ideal $\langle 1+ab \rangle$ is a prime ideal
        \item The ideal $\langle a \rangle$ is not necessarily a maximal ideal
    \end{enumerate}
    \hfill (GATE MA 2013)
%7
    \item Let X be a compact Hausdorff topological space and let Y be a topological space. Let $f: X \rightarrow Y$ be a bijective continuous mapping. Which of the following is TRUE?
    \begin{enumerate}
        \item f is a closed map but not necessarily an open map
        \item f is an open map but not necessarily a closed map
        \item f is both an open map and a closed map
        \item f need not be an open map or a closed map
    \end{enumerate}
    \hfill (GATE MA 2013)
%8
    \item Consider the linear programming problem: \\
    Maximize $x + \frac{3}{2}y$ \\
    subject to \\
    $2x + 3y \le 16,$ \\
    $x + 4y \le 18,$ \\
    $x \ge 0, y \ge 0$. \\
    If S denotes the set of all solutions of the above problem, then
    \begin{enumerate}
    \begin{multicols}{2}
        \item S is empty
        \item S is a singleton
        \item S is a line segment
        \item S has positive area
    \end{multicols}
    \end{enumerate}
    \hfill (GATE MA 2013)
%9
    \item Which of the following groups has a proper subgroup that is NOT cyclic?
    \begin{enumerate}
        \item $\mathbb{Z}_{15} \times \mathbb{Z}_{77}$
        \item $S_3$
        \item $(\mathbb{Z}, +)$
        \item $(\mathbb{Q}, +)$
    \end{enumerate}
    \hfill (GATE MA 2013)
%10
    \item The value of the integral $\int_{0}^{\infty} \int_{x}^{\infty} \left(\frac{1}{y}\right) e^{-y/2} dy dx$ is \underline{\hspace{1cm}}.
%11
    \item Suppose the random variable U has uniform distribution on [0,1] and $X = -2 \log U$. The density of X is
    \begin{enumerate}
        \item $f(x) = \begin{cases} e^{-x} & \text{if } x>0 \\ 0 & \text{otherwise} \end{cases}$
        \item $f(x) = \begin{cases} 2e^{-2x} & \text{if } x>0 \\ 0 & \text{otherwise} \end{cases}$
        \item $f(x) = \begin{cases} \frac{1}{2}e^{-x/2} & \text{if } x>0 \\ 0 & \text{otherwise} \end{cases}$
        \item $f(x) = \begin{cases} 1/2 & \text{if } x \in [0,2] \\ 0 & \text{otherwise} \end{cases}$
    \end{enumerate}
    \hfill (GATE MA 2013)
%12
    \item Let f be an entire function on $\mathbb{C}$ such that $|f(z)| \le 100 \log|z|$ for each z with $|z| \ge 2$. If $f(i) = 2i$, then $f(1)$
    \begin{enumerate}
    \begin{multicols}{2}
        \item must be 2
        \item must be 2i
        \item must be i
        \item cannot be determined from the given data
    \end{multicols}
    \end{enumerate}
    \hfill (GATE MA 2013)
%13
    \item The number of group homomorphisms from $\mathbb{Z}_3$ to $\mathbb{Z}_9$ is \underline{\hspace{1cm}}.
    \hfill (GATE MA 2013)
%14
    \item Let $u(x,t)$ be the solution to the wave equation $\frac{\partial^2 u}{\partial x^2}(x,t) = \frac{\partial^2 u}{\partial t^2}(x,t)$, with $u(x,0) = \cos(5\pi x)$, $\frac{\partial u}{\partial t}(x,0) = 0$. Then, the value of $u(1,1)$ is \underline{\hspace{1cm}}.
    \hfill (GATE MA 2013)
%15
    \item Let $f(x) = \sum_{n=1}^{\infty} \frac{\sin(nx)}{n^2}$. Then
    \begin{enumerate}
    \begin{multicols}{2}
        \item $\lim_{x \to 0} f(x) = 0$
        \item $\lim_{x \to 0} f(x) = 1$
        \item $\lim_{x \to 0} f(x) = \pi^2/6$
        \item $\lim_{x \to 0} f(x)$ does not exist
    \end{multicols}
    \end{enumerate}
    \hfill (GATE MA 2013)
%16
    \item Suppose X is a random variable with $P(X=k) = (1-p)^k p$ for $k \in \{0,1,2,...\}$ and some $p \in (0,1)$. For the hypothesis testing problem $$H_0: p = \frac{1}{2} H_1: p \neq \frac{1}{2}$$, consider the test "Reject $H_0$ if $X \leq A$ or if $X \geq B$" where A $<$ B are given positive integers. The type-I error of this test is
    \begin{enumerate}
        \item $1 + 2^{-B} - 2^{-A}$
        \item $1 - 2^{-B} + 2^{-A}$
        \item $1 + 2^{-B} - 2^{-A-1}$
        \item $1 - 2^{-B} + 2^{-A-1}$
    \end{enumerate}
    \hfill (GATE MA 2013)
%17
    \item Let G be a group of order 231. The number of elements of order 11 in G \\ is \underline{\hspace{1cm}}.
    \hfill (GATE MA 2013)
%18
    \item Let $f: \mathbb{R}^2 \rightarrow \mathbb{R}^2$ be defined by $f(x,y) = (e^{x+y}, e^{x-y})$. The area of the image of the region $\{(x,y) \in \mathbb{R}^2 : 0 < x, y < 1\}$ under the mapping f is
    \begin{enumerate}
    \begin{multicols}{4}
        \item 1
        \item $e-1$
        \item $e^2$
        \item $e^2 - 1$
    \end{multicols}
    \end{enumerate}
     \hfill (GATE MA 2013)
%19
    \item Which of the following is a field?
    \begin{enumerate}
        \item $\mathbb{C}[x] / \langle x^2+2 \rangle$
        \item $\mathbb{Z}[x] / \langle x^2+2 \rangle$
        \item $\mathbb{Q}[x] / \langle x^2-2 \rangle$
        \item $\mathbb{R}[x] / \langle x^2-2 \rangle$
    \end{enumerate}
    \hfill (GATE MA 2013)
%20
    \item Let $x_0 = 0$. Define $x_{n+1} = \cos x_n$ for every $n \ge 0$. Then
    \begin{enumerate}
        \item $\{x_n\}$ is increasing and convergent
        \item $\{x_n\}$ is decreasing and convergent
        \item $\{x_n\}$ is convergent and $x_{2n} < \lim_{m \to \infty} x_m < x_{2n+1}$ for every $n \in \mathbb{N}$
        \item $\{x_n\}$ is not convergent
    \end{enumerate}
    \hfill (GATE MA 2013)
%21
    \item Let C be the contour $|z|=2$ oriented in the anti-clockwise direction. The value of the integral $\oint_C z e^{3/z} dz$ is
    \begin{enumerate}
    \begin{multicols}{4}
        \item $3\pi i$
        \item $5\pi i$
        \item $7\pi i$
        \item $9\pi i$
    \end{multicols}
    \end{enumerate}
     \hfill (GATE MA 2013)
%22
    \item For each $\lambda > 0$, let $X_{\lambda}$ be a random variable with exponential density $\lambda e^{-\lambda x}$ on $(0, \infty)$. Then, $Var(\log X_{\lambda})$
    \begin{enumerate}
        \item is strictly increasing in $\lambda$
        \item is strictly decreasing in $\lambda$
        \item does not depend on $\lambda$
        \item first increases and then decreases in $\lambda$
    \end{enumerate}
    \hfill (GATE MA 2013)
%23
    \item Let $\{a_n\}$ be the sequence of consecutive positive solutions of the equation $\tan x = x$ and let $\{b_n\}$ be the sequence of consecutive positive solutions of the equation $\tan \sqrt{x} = x$. Then
    \begin{enumerate}
    \begin{multicols}{2}
        \item $\sum_{n=1}^{\infty} \frac{1}{a_n}$ converges but $\sum_{n=1}^{\infty} \frac{1}{b_n}$ diverges
        \item $\sum_{n=1}^{\infty} \frac{1}{a_n}$ diverges but $\sum_{n=1}^{\infty} \frac{1}{b_n}$ converges
        \item Both $\sum_{n=1}^{\infty} \frac{1}{a_n}$ and $\sum_{n=1}^{\infty} \frac{1}{b_n}$ converge
        \item Both $\sum_{n=1}^{\infty} \frac{1}{a_n}$ and $\sum_{n=1}^{\infty} \frac{1}{b_n}$ diverge
    \end{multicols}
    \end{enumerate}
    \hfill (GATE MA 2013)
%24
    \item Let f be an analytic function on $\overline{D} = \{z \in \mathbb{C} : |z| \le 1\}$. Assume that $|f(z)| \le 1$ for each $z \in \overline{D}$. Then, which of the following is NOT a possible value of $(e^f)''(0)$?
    \begin{enumerate}
    \begin{multicols}{4}
        \item 2
        \item 6
        \item $\frac{7}{9} e^{1/9}$
        \item $\sqrt{2} + i\sqrt{2}$
    \end{multicols}
    \end{enumerate}
    \hfill (GATE MA 2013)
%25
    \item The number of non-isomorphic abelian groups of order 24 is \underline{\hspace{1cm}}.
    \hfill (GATE MA 2013)
\end{enumerate}

\section*{Q.26 - Q.55 carry two marks each.}
\begin{enumerate}
    \setcounter{enumi}{25}
    %26
    \item Let V be the real vector space of all polynomials in one variable with real coefficients and having degree at most 20. Define the subspaces \\
    $W_1 = \cbrak{p \in V : p(1)=0, p(1/2)=0, p(5)=0, p(7)=0}$, \\
    $W_2 = \cbrak{p \in V : p(1/2)=0, p(3)=0, p(4)=0, p(7)=0}$. \\
    Then the dimension of $W_1 \cap W_2$ is \underline{\hspace{1cm}}.
    \hfill (GATE MA 2013)
%27
    \item Let $f, g: [0,1] \rightarrow \mathbb{R}$ be defined by \\
    $f(x) = \begin{cases} x & \text{if } x = \frac{1}{n} \text{ for } n \in \mathbb{N} \\ 0 & \text{otherwise} \end{cases}$ and $g(x) = \begin{cases} 1 & \text{if } x \in \mathbb{Q} \cap [0,1] \\ 0 & \text{otherwise} \end{cases}$. \\
    Then
    \begin{enumerate}
        \item Both f and g are Riemann integrable
        \item f is Riemann integrable and g is Lebesgue integrable
        \item g is Riemann integrable and f is Lebesgue integrable
        \item Neither f nor g is Riemann integrable
    \end{enumerate}
    \hfill (GATE MA 2013)
%28
    \item Consider the following linear programming problem: \\
    Maximize $x+3y+6z-w$ \\
    subject to \\
    $5x+y+6z+7w \le 20$, \\
    $x+2y+2z+9w \le 40$, \\
    $x \ge 0, y \ge 0, z \ge 0, w \ge 0$. \\
    Then the optimal value is \underline{\hspace{1cm}}.
    \hfill (GATE MA 2013)
%29
    \item Suppose X is a real-valued random variable. Which of the following values CANNOT be attained by $E[X]$ and $E[X^2]$ respectively?
    \begin{enumerate}
    \begin{multicols}{4}
        \item 0 and 1
        \item 2 and 3
        \item $\frac{1}{2}$ and $\frac{1}{3}$
        \item 2 and 5
    \end{multicols}
    \end{enumerate}
    \hfill (GATE MA 2013)
%30
    \item Which of the following subsets of $\mathbb{R}^2$ is NOT compact?
    \begin{enumerate}
        \item $\{(x,y) \in \mathbb{R}^2 : 1 \le x \le 1, y = \sin x\}$
        \item $\{(x,y) \in \mathbb{R}^2 : -1 \le y \le 1, y = x^8 - x^3 - 1\}$
        \item $\{(x,y) \in \mathbb{R}^2 : y=0, \sin(e^x)=0\}$
        \item $\{(x,y) \in \mathbb{R}^2 : x>0, y=\sin(\frac{1}{x})\} \cap \{(x,y) \in \mathbb{R}^2 : x>0, y=\frac{1}{x}\}$
    \end{enumerate}
    \hfill (GATE MA 2013)
%31
    \item Let M be the real vector space of $2 \times 3$ matrices with real entries. Let $T: M \rightarrow M$ be defined by \\
    $T\brak{\myvec{ x_1 & x_2 & x_3 \\ x_4 & x_5 & x_6 }} = \myvec{ -x_6 & x_4 & x_1 \\ x_3 & x_5 & x_2 }$. \\
    The determinant of T is \underline{\hspace{1cm}}.
    \hfill (GATE MA 2013)
%32
    \item Let H be a Hilbert space and let $\{e_n : n \ge 1\}$ be an orthonormal basis of H. Suppose $T: H \rightarrow H$ is a bounded linear operator. Which of the following CANNOT be true?
    \begin{enumerate}
        \item $T(e_n) = e_1$ for all $n \ge 1$
        \item $T(e_n) = e_{n+1}$ for all $n \ge 1$
        \item $T(e_n) = \sqrt{\frac{n+1}{n}} e_n$ for all $n \ge 1$
        \item $T(e_n) = e_{n-1}$ for all $n \ge 2$ and $T(e_1) = 0$
    \end{enumerate}
    \hfill (GATE MA 2013)
%33
    \item The value of the limit $\lim_{n \to \infty} \frac{2^{-n^2}}{\sum_{k=n+1}^{\infty} 2^{-k^2}}$ is
    \begin{enumerate}
    \begin{multicols}{4}
        \item 0
        \item some $c \in (0,1)$
        \item 1
        \item $\infty$
    \end{multicols}
    \end{enumerate}
    \hfill (GATE MA 2013)
%34
    \item Let $f: \mathbb{C} \setminus \{3i\} \rightarrow \mathbb{C}$ be defined by $f(z) = \frac{z-i}{iz+3}$. Which of the following statements about f is FALSE?
    \begin{enumerate}
        \item f is conformal on $\mathbb{C} \setminus \{3i\}$
        \item f maps circles in $\mathbb{C} \setminus \{3i\}$ onto circles in $\mathbb{C}$
        \item All the fixed points of f are in the region $\{z \in \mathbb{C} : Im(z) > 0\}$
        \item There is no straight line in $\mathbb{C} \setminus \{3i\}$ which is mapped onto a straight line in $\mathbb{C}$ by f
    \end{enumerate}
    \hfill (GATE MA 2013)
%35
    \item The matrix $A = \myvec{ 1 & 2 & 0 \\ 1 & 3 & 1 \\ 0 & 1 & 3 }$ can be decomposed uniquely into the product $A=LU$, where $L = \myvec{ 1 & 0 & 0 \\ l_{21} & 1 & 0 \\ l_{31} & l_{32} & 1 }$ and $U = \myvec{ u_{11} & u_{12} & u_{13} \\ 0 & u_{22} & u_{23} \\ 0 & 0 & u_{33} }$. The solution of the system $LX = \myvec{1 & 2 & 2}^t$ is
    \begin{enumerate}
    \begin{multicols}{4}
        \item $\myvec{1&1&1}^t$
        \item $\myvec{1&1 &0}^t$
        \item $\myvec{0 &1& 1}^t$
        \item $\myvec{1 &0& 1}^t$
    \end{multicols}
    \end{enumerate}
    \hfill (GATE MA 2013)
    
%36
    \item Let $S = \{x \in \mathbb{R} : x \ge 0, \sum_{n=1}^{\infty} x^{\sqrt{n}} < \infty\}$. Then the supremum of S is
    \begin{enumerate}
    \begin{multicols}{4}
        \item 1
        \item $\frac{1}{e}$
        \item 0
        \item $\infty$
    \end{multicols}
    \end{enumerate}
    \hfill (GATE MA 2013)
%37
    \item The image of the region $\{z \in \mathbb{C} : Re(z) > Im(z) > 0\}$ under the mapping $z \mapsto e^{z^2}$ is
    \begin{enumerate}
    \begin{multicols}{2}
        \item $\{w \in \mathbb{C} : Re(w) > 0, Im(w) > 0\}$
        \item $\{w \in \mathbb{C} : Re(w) > 0, Im(w) > 0, |w| > 1\}$
        \item $\{w \in \mathbb{C} : |w| > 1\}$
        \item $\{w \in \mathbb{C} : Im(w) > 0, |w| > 1\}$
    \end{multicols}
    \end{enumerate}
    \hfill (GATE MA 2013)
%38
    \item Which of the following groups contains a unique normal subgroup of order four?
    \begin{enumerate}
    \begin{multicols}{2}
        \item $\mathbb{Z}_2 \oplus \mathbb{Z}_4$
        \item The dihedral group, $D_4$, of order eight
        \item The quaternion group, $Q_8$
        \item $\mathbb{Z}_2 \oplus \mathbb{Z}_2 \oplus \mathbb{Z}_2$
    \end{multicols}
    \end{enumerate}
    \hfill (GATE MA 2013)
%39
    \item Let B be a real symmetric positive-definite $n \times n$ matrix. Consider the inner product on $\mathbb{R}^n$ defined by $\langle X, Y \rangle = Y^t BX$. Let A be an $n \times n$ real matrix and let $T: \mathbb{R}^n \rightarrow \mathbb{R}^n$ be the linear operator defined by $T(X) = AX$ for all $X \in \mathbb{R}^n$. If S is the adjoint of T, then $S(X) = CX$ for all $X \in \mathbb{R}^n$, where C is the matrix
    \begin{enumerate}
    \begin{multicols}{4}
        \item $B^{-1}A^t B$
        \item $B A^t B^{-1}$
        \item $B^{-1}AB$
        \item $A^t$
    \end{multicols}
    \end{enumerate}
    \hfill (GATE MA 2013)
%40
    \item Let X be an arbitrary random variable that takes values in $\{0, 1, ..., 10\}$. The minimum and maximum possible values of the variance of X are
    \begin{enumerate}
    \begin{multicols}{4}
        \item 0 and 30
        \item 1 and 30
        \item 0 and 25
        \item 1 and 25
    \end{multicols}
    \end{enumerate}
    \hfill (GATE MA 2013)
%41
    \item Let M be the space of all $4 \times 3$ matrices with entries in the finite field of three elements. Then the number of matrices of rank three in M is
    \begin{enumerate}
        \item $(3^4-3)(3^4-3^2)(3^4-3^3)$
        \item $(3^4-1)(3^4-2)(3^4-3)$
        \item $(3^4-1)(3^4-3)(3^4-3^2)$
        \item $3^4(3^4-1)(3^4-2)$
    \end{enumerate}
    \hfill (GATE MA 2013)
%42
    \item Let $V$ be a vector space of dimension $m \ge 2$. Let $T: V \rightarrow V$ be a linear transformation such that $T^{n+1}=0$ and $T^n \neq 0$ for some $n \ge 1$. Then which of the following is necessarily TRUE?
    \begin{enumerate}
    \begin{multicols}{2}
        \item Rank$(T^n) \le$ Nullity$(T^n)$
        \item trace$(T) \neq 0$
        \item T is diagonalizable
        \item $n=m$
    \end{multicols}
    \end{enumerate}
    \hfill (GATE MA 2013)
%43
    \item Let X be a convex region in the plane bounded by straight lines. Let X have 7 vertices. Suppose $f(x,y) = ax+by+c$ has maximum value M and minimum value N on X and $N < M$. Let $S = \{P : P \text{ is a vertex of X and } N < f(P) < M\}$. If S has n elements, then which of the following statements is TRUE?
    \begin{enumerate}
    \begin{multicols}{4}
        \item n cannot be 5
        \item n can be 2
        \item n cannot be 3
        \item n can be 4
    \end{multicols}
    \end{enumerate}
    \hfill (GATE MA 2013)
%44
    \item Which of the following statements are TRUE? \\
    P: If $f \in L^1(\mathbb{R})$, then f is continuous. \\
    Q: If $f \in L^1(\mathbb{R})$ and $\lim_{|x| \to \infty} f(x)$ exists, then the limit is zero. \\
    R: If $f \in L^1(\mathbb{R})$, then f is bounded. \\
    S: If $f \in L^1(\mathbb{R})$ is uniformly continuous, then $\lim_{|x| \to \infty} f(x)$ exists and equals zero.
    \begin{enumerate}
    \begin{multicols}{4}
        \item Q and S only
        \item P and R only
        \item P and Q only
        \item R and S only
    \end{multicols}
    \end{enumerate}
    \hfill (GATE MA 2013)
%45
    \item Let $u$ be a real valued harmonic function on $\mathbb{C}$. Let $g: \mathbb{R}^2 \rightarrow \mathbb{R}$ be defined by $g(x,y) = \frac{1}{2\pi} \int_0^{2\pi} u(e^{i\theta}(x+iy)) \sin\theta d\theta$. Which of the following statements is TRUE?
    \begin{enumerate}
        \item g is a harmonic polynomial
        \item g is a polynomial but not harmonic
        \item g is harmonic but not a polynomial
        \item g is neither harmonic nor a polynomial
    \end{enumerate}
    \hfill (GATE MA 2013)
%46
    \item Let $S = \{z \in \mathbb{C} : |z|=1\}$ with the induced topology from $\mathbb{C}$ and let $f: [0,2] \rightarrow S$ be defined as $f(t) = e^{2\pi i t}$. Then, which of the following is TRUE?
    \begin{enumerate}
        \item K is closed in [0,2] $\implies f(K)$ is closed in S
        \item U is open in [0,2] $\implies f(U)$ is open in S
        \item $f(X)$ is closed in S $\implies X$ is closed in [0,2]
        \item $f(Y)$ is open in S $\implies Y$ is open in [0,2]
    \end{enumerate}
    \hfill (GATE MA 2013)
%47
    \item Assume that all the zeros of the polynomial $a_n x^n + a_{n-1}x^{n-1} + \dots + a_1 x + a_0$ have negative real parts. If $u(t)$ is any solution to the ordinary differential equation \\
    $a_n \frac{d^n u}{dt^n} + a_{n-1} \frac{d^{n-1} u}{dt^{n-1}} + \dots + a_1 \frac{du}{dt} + a_0 u = 0$, \\
    then $\lim_{t \to \infty} u(t)$ is equal to
    \begin{enumerate}
    \begin{multicols}{4}
        \item 0
        \item 1
        \item $n-1$
        \item $\infty$
    \end{multicols}
    \end{enumerate}
    \hfill (GATE MA 2013)
    
\end{enumerate}
\subsection*{Common Data for Questions 48 and 49}
Let $c_{00}$ be the vector space of all complex sequences having finitely many non-zero terms. Equip $c_{00}$ with the inner product $\langle x, y \rangle = \sum_{n=1}^{\infty} x_n \overline{y_n}$ for all $x=(x_n)$ and $y=(y_n)$ in $c_{00}$. Define $f: c_{00} \rightarrow \mathbb{C}$ by $f(x) = \sum_{n=1}^{\infty} \frac{x_n}{n}$. Let N be the kernel of f.
\begin{enumerate}
    \setcounter{enumi}{47}
%48
    \item Which of the following is FALSE?
    \begin{enumerate}
        \item f is a continuous linear functional
        \item $\|f\| \le \frac{\pi}{\sqrt{6}}$
        \item There does not exist any $y \in c_{00}$ such that $f(x) = \langle x, y \rangle$ for all $x \in c_{00}$
        \item $N^{\perp} \neq \{0\}$
    \end{enumerate}
    \hfill (GATE MA 2013)
%49
    \item Which of the following is FALSE?
    \begin{enumerate}
        \item $c_{00} \neq N$
        \item N is closed
        \item $c_{00}$ is not a complete inner product space
        \item $c_{00} = N \oplus N^{\perp}$
    \end{enumerate}
    \hfill (GATE MA 2013)
\end{enumerate}
\subsection*{Common Data for Questions 50 and 51}
Let $X_1, X_2, \dots, X_n$ be an i.i.d. random sample from exponential distribution with mean $\mu$. In other words, they have density $f(x) = \begin{cases} \frac{1}{\mu} e^{-x/\mu} & \text{if } x>0 \\ 0 & \text{otherwise} \end{cases}$.
\begin{enumerate}
    \setcounter{enumi}{49}
%50
    \item Which of the following is NOT an unbiased estimate of $\mu$?
    \begin{enumerate}
        \item $X_1$
        \item $\frac{1}{n-1}\brak{X_2 + X_3 + \dots + X_n}$
        \item $n \cdot \brak{\min\{X_1, X_2, \dots, X_n\}}$
        \item $\frac{1}{n} \max\{X_1, X_2, \dots, X_n\}$
    \end{enumerate}
    \hfill (GATE MA 2013)
%51
    \item Consider the problem of estimating $\mu$. The m.s.e (mean square error) of the estimate $T(X) = \frac{X_1 + X_2 + \dots + X_n}{n+1}$ is
    \begin{enumerate}
    \begin{multicols}{4}
        \item $\mu^2$
        \item $\frac{1}{n+1} \mu^2$
        \item $\frac{1}{(n+1)^2} \mu^2$
        \item $\frac{n^2}{(n+1)^2} \mu^2$
    \end{multicols}
    \end{enumerate}
    \hfill (GATE MA 2013)
\end{enumerate}
\subsection*{Statement for Linked Answer Questions 52 and 53}
Let $X = \{(x,y) \in \mathbb{R}^2 : x^2+y^2=1\} \cup ([-1,1] \times \{0\}) \cup (\{0\} \times [-1,1])$.
Let $n_0 = \max\{k : k < \infty, \text{ there are } k \text{ distinct points } p_1, \dots, p_k \in X \text{ such that } X \setminus \{p_1, \dots, p_k\} \text{ is connected}\}$.
\begin{enumerate}
    \setcounter{enumi}{51}
%52
    \item The value of $n_0$ is \underline{\hspace{1cm}}.
    \hfill (GATE MA 2013)
    
%53
    \item Let $q_1, \dots, q_{n_0+1}$ be $n_0+1$ distinct points and $Y = X \setminus \{q_1, \dots, q_{n_0+1}\}$. Let m be the number of connected components of Y. The maximum possible value of m is \underline{\hspace{1cm}}.
\end{enumerate}
\hfill (GATE MA 2013)
\subsection*{Statement for Linked Answer Questions 54 and 55}
Let $W(y_1, y_2)$ be the Wronskian of two linearly independent solutions $y_1$ and $y_2$ of the equation $y'' + P(x)y' + Q(x)y = 0$.
\begin{enumerate}
    \setcounter{enumi}{53}
%54
    \item The product $W(y_1, y_2)P(x)$ equals
    \begin{enumerate}
    \begin{multicols}{2}
        \item $y_2 y_1'' - y_1 y_2''$
        \item $y_1 y_2' - y_2 y_1'$
        \item $y_1' y_2'' - y_2' y_1''$
        \item $y_2' y_1' - y_1'' y_2''$
    \end{multicols}
    \end{enumerate}
    \hfill (GATE MA 2013)
%55
    \item If $y_1 = e^{2x}$ and $y_2 = x e^{2x}$, then the value of $P(0)$ is
    \begin{enumerate}
    \begin{multicols}{4}
        \item 4
        \item -4
        \item 2
        \item -2
    \end{multicols}
    \end{enumerate}
    \hfill (GATE MA 2013)
\end{enumerate}
\section*{General Aptitude (GA) Questions}
\subsection*{Q. 56 – Q. 60 carry one mark each.}
\begin{enumerate}
    \setcounter{enumi}{55}
%56
    \item A number is as much greater than 75 as it is smaller than 117. The number is:
    \begin{enumerate}
    \begin{multicols}{4}
        \item 91
        \item 93
        \item 89
        \item 96
    \end{multicols}
    \end{enumerate}
    \hfill (GATE MA 2013)
%57
    \item The professor \underline{ordered to} the students to go out of the class. \\
    \hspace*{2cm} I \hspace{1.2cm} II \hspace{1.3cm} III \hspace{1.8cm} IV \\
    Which of the above underlined parts of the sentence is grammatically incorrect?
    \begin{enumerate}
    \begin{multicols}{4}
        \item I
        \item II
        \item III
        \item IV
    \end{multicols}
    \end{enumerate}
    \hfill (GATE MA 2013)
%58
    \item Which of the following options is the closest in meaning to the word given below: \\
    \textbf{Primeval}
    \begin{enumerate}
    \begin{multicols}{2}
        \item Modern
        \item Historic
        \item Primitive
        \item Antique
    \end{multicols}
    \end{enumerate}
    \hfill (GATE MA 2013)
%59
    \item Friendship, no matter how \underline{\hspace{1cm}} it is, has its limitations.
    \begin{enumerate}
        \item cordial
        \item intimate
        \item secret
        \item pleasant
    \end{enumerate}
    \hfill (GATE MA 2013)
%60
    \item Select the pair that best expresses a relationship similar to that expressed in the pair: \\
    \textbf{Medicine: Health}
    \begin{enumerate}
    \begin{multicols}{2}
        \item Science: Experiment
        \item Wealth: Peace
        \item Education: Knowledge
        \item Money: Happiness
    \end{multicols}
    \end{enumerate}
    \hfill (GATE MA 2013)
\end{enumerate}
\subsection*{Q. 61 to Q. 65 carry two marks each.}
\begin{enumerate}
    \setcounter{enumi}{60}
%61
    \item X and Y are two positive real numbers such that $2X + Y \le 6$ and $X + 2Y \le 8$. For which of the following values of (X, Y) the function $f(X, Y) = 3X + 6Y$ will give maximum value?
    \begin{enumerate}
        \item (4/3, 10/3)
        \item (8/3, 20/3)
        \item (8/3, 10/3)
        \item (4/3, 20/3)
    \end{enumerate}
    \hfill (GATE MA 2013)
%62
    \item If $|4X - 7| = 5$ then the values of $2|X| - |-X|$ is:
    \begin{enumerate}
    \begin{multicols}{4}
        \item 2, 1/3
        \item 1/2, 3
        \item 3/2, 9
        \item 2/3, 9
    \end{multicols}
    \end{enumerate}
    \hfill (GATE MA 2013)
%63
    \item Following table provides figures (in rupees) on annual expenditure of a firm for two years - 2010 and 2011.
    \begin{center}
        \begin{tabular}{|l|c|c|}
            \hline
            \textbf{Category} & \textbf{2010} & \textbf{2011} \\
            \hline
            Raw material & 5200 & 6240 \\
            Power \& fuel & 7000 & 9450 \\
            Salary \& wages & 9000 & 12600 \\
            Plant \& machinery & 20000 & 25000 \\
            Advertising & 15000 & 19500 \\
            Research \& Development & 22000 & 26400 \\
            \hline
        \end{tabular}
    \end{center}
    In 2011, which of the following two categories have registered increase by same percentage?
    \begin{enumerate}
        \item Raw material and Salary \& wages
        \item Salary \& wages and Advertising
        \item Power \& fuel and Advertising
        \item Raw material and Research \& Development
    \end{enumerate}
    \hfill (GATE MA 2013)
%64
    \item A firm is selling its product at Rs. 60 per unit. The total cost of production is Rs. 100 and firm is earning total profit of Rs. 500. Later, the total cost increased by 30\%. By what percentage the price should be increased to maintained the same profit level.
    \begin{enumerate}
    \begin{multicols}{4}
        \item 5
        \item 10
        \item 15
        \item 30
    \end{multicols}
    \end{enumerate}
    \hfill (GATE MA 2013)
%65
    \item Abhishek is elder to Savar. \\
    Savar is younger to Anshul. \\
    Which of the given conclusions is logically valid and is inferred from the above statements?
    \begin{enumerate}
        \item Abhishek is elder to Anshul
        \item Anshul is elder to Abhishek
        \item Abhishek and Anshul are of the same age
        \item No conclusion follows
    \end{enumerate}
    \hfill (GATE MA 2013)
\end{enumerate}

\begin{center}
\textbf{END OF QUESTION PAPER}
\end{center}

\end{document}
