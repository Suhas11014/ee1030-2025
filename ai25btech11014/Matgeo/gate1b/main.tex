\documentclass{beamer}
\usepackage[utf8]{inputenc}
\usepackage{amsmath, amssymb}
\usetheme{Madrid}
\usecolortheme{default}

\title{Counting Linearly Independent Eigenvectors}
\author{Gooty Suhas}
\date{}

\begin{document}

\frame{\titlepage}

\begin{frame}{Question}
Find the number of linearly independent eigenvectors of the matrix:
\[
\vec{A} =
\begin{bmatrix}
2 & 2 & 0 & 0 \\
2 & 1 & 0 & 0 \\
0 & 0 & 3 & 0 \\
0 & 0 & 0 & 4
\end{bmatrix}
\]
\end{frame}

\begin{frame}{Step 1: Block Diagonal Structure}
Matrix \( \vec{A} \) is block diagonal:
\[
\vec{A} =
\begin{bmatrix}
\vec{B} & 0 \\
0 & \vec{C}
\end{bmatrix}
\quad \text{where}
\quad
\vec{B} = \begin{bmatrix} 2 & 2 \\ 2 & 1 \end{bmatrix},\quad
\vec{C} = \begin{bmatrix} 3 & 0 \\ 0 & 4 \end{bmatrix}
\]

Eigenvalues of \( \vec{A} \) are the union of eigenvalues of \( \vec{B} \) and \( \vec{C} \).
\end{frame}

\begin{frame}{Step 2: Discriminant Test for \( \vec{B} \)}
For any \( 2 \times 2 \) matrix \( \begin{bmatrix} a & b \\ c & d \end{bmatrix} \), the characteristic polynomial is:
\[
\lambda^2 - \text{Tr}(\vec{B})\lambda + \det(\vec{B})
\]
Discriminant:
\[
\Delta = \text{Tr}(\vec{B})^2 - 4\det(\vec{B})
\]

Apply to \( \vec{B} = \begin{bmatrix} 2 & 2 \\ 2 & 1 \end{bmatrix} \):
\[
\text{Tr}(\vec{B}) = 2 + 1 = 3,\quad \det(\vec{B}) = 2 \cdot 1 - 2 \cdot 2 = -2
\]
\[
\Delta = 3^2 - 4(-2) = 9 + 8 = 17 > 0
\]

So \( \vec{B} \) has two distinct real eigenvalues.
\end{frame}

\begin{frame}{Step 3: Eigenvalues of \( \vec{C} \)}
Since \( \vec{C} \) is diagonal:
\[
\text{Eigenvalues of } \vec{C} = 3,\quad 4
\]
Each contributes one linearly independent eigenvector.
\end{frame}

\begin{frame}{Step 4: Final Count}
\begin{itemize}
    \item \( \vec{B} \): 2 distinct eigenvalues → 2 independent eigenvectors
    \item \( \vec{C} \): 2 distinct eigenvalues → 2 independent eigenvectors
\end{itemize}
\[
\text{Total linearly independent eigenvectors} = 2 + 2 = \boxed{4}
\]
\end{frame}

\begin{frame}{Conclusion}
Therefore, the matrix
\[
\vec{A} =
\begin{bmatrix}
2 & 2 & 0 & 0 \\
2 & 1 & 0 & 0 \\
0 & 0 & 3 & 0 \\
0 & 0 & 0 & 4
\end{bmatrix}
\]
has \textbf{4 distinct and non-linearly dependent eigenvectors}.
\end{frame}

\begin{frame}{Additional Proof: Block Diagonal Eigenvalues}
Let
\[
\vec{A} =
\begin{bmatrix}
\vec{A}_1 & 0 \\
0 & \vec{A}_2
\end{bmatrix}
\quad \text{where } \vec{A}_1 \in \mathbb{R}^{m \times m},\ \vec{A}_2 \in \mathbb{R}^{(n-m) \times (n-m)}
\]

Suppose \( \vec{x}_1 \in \mathbb{R}^m \) is an eigenvector of \( \vec{A}_1 \) with eigenvalue \( \lambda \).  
Define \( \vec{x} = \begin{bmatrix} \vec{x}_1 \\ \vec{0} \end{bmatrix} \in \mathbb{R}^n \)

Then:
\[
\vec{A} \vec{x} =
\begin{bmatrix}
\vec{A}_1 \vec{x}_1 \\
\vec{0}
\end{bmatrix}
=
\begin{bmatrix}
\lambda \vec{x}_1 \\
\vec{0}
\end{bmatrix}
= \lambda \vec{x}
\]

So \( \vec{x} \) is an eigenvector of \( \vec{A} \) with eigenvalue \( \lambda \).  
The same holds for eigenvectors of \( \vec{A}_2 \).

\textbf{Conclusion:} Eigenvalues of \( \vec{A} \) are the union of eigenvalues of its diagonal blocks.
\end{frame}

\end{document}